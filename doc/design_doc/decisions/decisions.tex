\documentclass[../design_doc.tex]{subfiles}

\begin{document}

\section{Decisions}\label{sec:dec}
    While not every decision is outlined in this section, decisions that were made unexpectedly or creatively are included.

    \subsection{TDS Seperate Implementation}\label{sec:dec:seperatetds}
        Ideally, this implementation of the TDS protocol could be bundled and imported into other server database projects. Thus, I have decided to seperate its implementation from the main Bookdex implementation. This is visible in the \repo{project repository} as the ``bookdex'' directory is seperate from the ``TabularDataStreamProtocol'' directory. Hopefully, this approach makes testing TDS easier as it will be modular.

    \subsection{Omitting Timers}\label{sec:dec:timers}
        As can be seen in Figure~\ref{fig:tdsuml}, various timers are used by the client to ensure the client is not waiting too long for a response from the server. It's important to note that TDS does not require these timers. Initial implementation will omit these timers for simplicity's and time's sake.

    \subsection{Client-side I/O}\label{sec:dec:clientio}
        As seen in Section~\ref{sec:classes}, the states in which a TDSClient can exist handle the Listen() operation. Initially, Listen() was intended for receiving packets send from the server to the client. However, when in the TDSClientLoggedInState, the user should be able to enter a request to be sent to the server before the client listens for a response. The definition of Listen() was then expanded to include ``listening'' for input from the user.

        Anticipating user input to come from various input streams, I planned to create a custom input stream to which all input could be written and from which the program could consistently read. Upon researching halting the program to wait for input on said stream, I began to fear I had gone astray. Instead, I opted to create custom I/O, namely, TDSIO (Tabular Data Stream Protocol Input/Output). Now, when a user writes to the ``input stream'', their input is enqueued to a queue of inputs. When these inputs are flushed, a notification of a complete task will be sent to the listening TDSClientLoggedInState. The queue is then emptied and the TDSClientLoggedInState can do with the inputs what it will. This way, implementation of a view, whether GUI or console, will be supported (I think).

\end{document}